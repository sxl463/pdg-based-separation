%  Mathpartir --- Math Paragraph for Typesetting Inference Rules
%
%  Copyright (C) 2001, 2002, 2003, 2005, 2015 Didier R�my
%
%  Author         : Didier Remy 
%  Version        : 1.3.1
%  Bug Reports    : to author
%  Web Site       : http://pauillac.inria.fr/~remy/latex/
% 
%  Mathpartir is free software; you can redistribute it and/or modify
%  it under the terms of the GNU General Public License as published by
%  the Free Software Foundation; either version 2, or (at your option)
%  any later version.
%  
%  Mathpartir is distributed in the hope that it will be useful,
%  but WITHOUT ANY WARRANTY; without even the implied warranty of
%  MERCHANTABILITY or FITNESS FOR A PARTICULAR PURPOSE.  See the
%  GNU General Public License for more details 
%  (http://pauillac.inria.fr/~remy/license/GPL).
%
%%%%%%%%%%%%%%%%%%%%%%%%%%%%%%%%%%%%%%%%%%%%%%%%%%%%%%%%%%%%%%%%%%%%%%%%%%%%
%  File mathpartir.tex (Documentation)
%%%%%%%%%%%%%%%%%%%%%%%%%%%%%%%%%%%%%%%%%%%%%%%%%%%%%%%%%%%%%%%%%%%%%%%%%%%%

\documentclass {article}

\usepackage {mathpartir}
\usepackage {listings}
\usepackage {array}
\usepackage {url}
\usepackage{caption}
\usepackage{amssymb}

\usepackage{tablefootnote}
\captionsetup[lstlisting]{position=bottom}
%\usepackage{courier}



\newif \ifhevea 
%HEVEA \heveatrue
\ifhevea 
\usepackage {hevea}
\fi

\lstset {basicstyle=\tt}


\let \lst \verb
%
%\title {
%HEVEA\begin{tabular}{>{\huge}c}
%\textbf{MATH} formulas in \\ 
%\textbf{PAR}ragraph mode\\[-1ex]
%\ifhevea\else{}\hskip 0em\hbox to 12em {\hrulefill}\\\fi
%HEVEA\hline
%\textbf Typesetting \textbf Inference \textbf Rules \\
%HEVEA\end{tabular}
%}


\title {Type Encoding/Decoding rules: from C to Protocol Buffer}




%\author {Didier R{\'{e}}my}
\author {Shen Liu}

\date {(Version 1.5, last modified 07/05/2016)}


\begin{document}

\maketitle

\begin{abstract}
Type conversion between C and protocol buffer is an important issue in our project. When the separation is done and the RPC tool begins working, we must automatically restore all the arguments for each RPC function in the receiver process, which means the function parameter types need to be transmitted between two processes fully and exactly. Unfortunately, the type system that protocol buffer supports is quite weak. To make our project automatically run in the end, we have to design a type conversion protocol to let protocol buffer automatically convert some advanced C types(e.g. pointer) into protocol buffer types. 

\end{abstract}


\section {Background}

We use gRPC, which is fully based on google protocol buffer, to deal with RPC issues in our project. In gRPC, a C type must be packed to protocol buffer ``\texttt{message}'' type in a .proto file(IDL file) for further transmission. For example, if you have a C function \texttt{int foo(int x)} which needs to be called remotely, then in your .proto file, the argument type \texttt{int} can be packed in protocol buffer as follows:\\
\begin{lstlisting}[xleftmargin=.1\textwidth]  % Start your code-block

message M{                    message M{
  int64 x=1;      or            int32 x=1
}                             }

\end{lstlisting}

Next, protocol buffer will automatically generate a group of read/write APIs for each message. Here is an API for x's value assignment:  

\begin{lstlisting}[xleftmargin=.1\textwidth]  % Start your code-block

 void set_x(::google::protobuf::int32 value);
\end{lstlisting}

and an API for getting the value of x:

\begin{lstlisting}[xleftmargin=.1\textwidth]  % Start your code-block

 inline ::google::protobuf::int32 M::x() const {
  return x_;
}
\end{lstlisting}


Here is a more complex C-protobuf type conversion sample:
\begin{lstlisting}[xleftmargin=.1\textwidth]  % Start your code-block

typedef struct{                  message Circle{
  int x;                           message Point{
  int y;                             int64 x=1;
}Point;                              int64 y=2;
                      --->         }  
typedef struct{                    double radius=1;
  Point center;                  }
  double radius;
}Circle;

\end{lstlisting}


Our project can automatically finish this conversion for all scalar types and simple composite types as ``Circle''. However, when parameter types become more and more complex, especially for those structures with multi-level pointers, generating a correct .proto file automatically as before will be a real challenge. To achieve this goal, a possible way is designing a type-conversion protocol to make our project work more intelligently. Simply speaking, for any C type input, first we use such a protocol to convert it into an integer array(encoding), and then construct the ``message'' type in .proto. On the receiver side, we do array parsing to restore the original C types(decoding) instead of parsing the complex .proto file.



\section {Type system and encoding/decoding rules}

In this draft we only use a small subset of C type system to show how the encoding/decoding idea works. Here is how our toy type system looks like:

\begin{lstlisting}[xleftmargin=.1\textwidth]  % Start your code-block

Type t := int | t* | struct {t1; t2; ... ;tn}
          | tname S
 
\end{lstlisting}

Any pair of form (\texttt{type,value}) based on this type system will be encoded as an array of bytes(\texttt{bytes[] lst}), and the first byte(see table 1) in this array denotes what type this array corresponds to. 





\begin{table}[ht]
% title of Table
\centering 
% used for centering table
\begin{tabular}{c c c c}
% centered columns (4 columns)
\hline                        %inserts double horizontal lines
lst[0] & type \\ [0.5ex]% inserts table 
%heading
\hline                  % inserts single horizontal line
0 & int \\% inserting body of the table
1 & pointer \\
2 & struct \{$t_1;t_2;...;t_n$\}\\
3 & tname S \\

\hline
%inserts single line
\end{tabular}
\label{table:nonlin}% is used to refer this table in the text
\caption{type mapping rules}
\end{table}

As we can see from Table 1, any encoding/decoding operation related to type \texttt{tname S} requires knowing the associative type \texttt{struct\{...\}} of \texttt{tname S}. In our framework, we use a name-type mapping table(e.g. table 2) to map each name string which represents a struct to its corresponding \texttt{struct} type.

\begin{table}[ht]
% title of Table
\centering 
% used for centering table
\begin{tabular}{c c c c}
% centered columns (4 columns)
\hline                        %inserts double horizontal lines
name & struct \\ [0.5ex]% inserts table 
%heading
\hline                  % inserts single horizontal line
S1 & struct \{int;int;\} \\% inserting body of the table
S2 & struct \{int;int*;\} \\
S3 & struct \{int;int*; struct\{int; int\}*;\}\\
... & ... \\

\hline
%inserts single line
\end{tabular}
\label{table:nonlin}% is used to refer this table in the text
\caption{A name--type mapping table example}
\end{table}

Besides, in each round for encoding, we also use an auxiliary table called pointer table to record each pointer value that ever appeared. By doing this we can identify some complex function arguments(e.g. circular linked list).\\


Once we have such auxiliary tables, we can easily construct the encoding/decoding rules for our type system as follows:

%C value --> bytes[]:
%---------------------------------------------------------------------
%bytes[] to C value:      
%---------------------------------------------------------------------      

%fold(struct {t1;...;tn}): map a structure in the shared mapping 
%						 table, return a tname S type                       


%\begin{lstlisting}[xleftmargin=.1\textwidth]
\begin{lstlisting}  % Start your code-block



Basic value conversion functions:

intToBytes(int): convert an integer to a byte string.

symbolToBytes(S): convert a symbol S to a byte string.

ptrToBytes(int): convert a pointer address(int) to a byte string.            
            
bytesToInt(bytes[]): convert a byte string to an integer.

bytesToSymbol(bytes[]): convert a byte string to a symbol.

bytesToPtr(bytes[]) convert a byte string to a hexadecimal integer.

dereference(int): return the value that a pointer points to.

getTypeFromTable(S): look up S in the mapping table and return its 
                     associative type.
								
								
Definition encode: (type,value) (t,v) -> bytes[]
  match t with
  | int    => 0::intToBytes(v)
  | t*     => 1::ptrToBytes(v)::encode(t,dereference(v)) 
  | struct {(t1,v1);...;(tn,vn)} 
           => 2::intToBytes(n)::encode(t1,v1)::...::encode(tn,vn)
  | tname S 
           => 3::symbolToBytes(S)::encode(getTypeFromTable(S),v)
  end. 
           

Definition decode bytes[] lst =
  match lst[0] with
    | 0 => ((int, bytesToInt(lst[1...4])), lst+5)  
      
    | 1 => let ((t1,dereference(v)), l1) = decode (lst+5) in ((t1*, v), l1)
           //v = bytesToPtr(lst[1...4])
           
    | 2 => let n = bytesToInt(lst[1]) in    		              
           let ((t1,v1),l1) = decode (lst+5) in
	       let ((t2,v2),l2) = decode l1 in
       	   ...
      	   let ((tn,vn),ln) = decode l_{n-1} in
      	   (struct {(t1,v1);(t2,v2);...;(tn,vn)}, ln)

    | 3 => let S = bytesToSymbol(lst[5...5+length(S)-1]) in
                               /*length(S) = bytesToInt(lst[1...4])*/
      	   decode(lst+offset)  /* offset = 1+4+length(S)*/
  end
            
\end{lstlisting}         
%    (search S in the struct name list, if found, break and return struct S)
    
 
Now consider a circular linked list example:

\begin{lstlisting}[xleftmargin=.1\textwidth]
typedef struct Node{
 int val; 
 Node* next;
}Node_t;

Node_t *head = (Node_t*) malloc(sizeof(Node_t)); //head: 0x0004
Node_t *tail = (Node_t*) malloc(sizeof(Node_t)); //tail: 0x0008

head->val = 10;
head->next = tail;

tail->val = 20;
tail->next = head; // circular linked list 

\end{lstlisting}


Assume that we want to send this circular linked list from sender to
receiver, then the encode/decode process is as follow:
%[xleftmargin=.1\textwidth]
\begin{lstlisting}
encode(Node_t *head, 0x0004)

= 1::ptrToBytes(0x0004)
   ::encode(Node_t, dereference(0x0004))
   /* dereference(0x0004) = {10,0x0008} */

   (pointer table: {0x0004})

= 1::ptrToBytes(0x0004)
   ::3::symbolToBytes(Node_t)
      ::encode(getTypeFromTable(Node_t),{10,0x0008})

   (pointer table: {0x0004})

= 1::ptrToBytes(0x0004)
   ::3::symbolToBytes(Node_t)
	  ::encode(struct Node{int, struct Node*},{10,0x0008})
   
   (pointer table: {0x0004})

= 1::ptrToBytes(0x0004)
   ::3::symbolToBytes(Node_t)
      ::2::intToBytes(2) /*two fields*/   
         ::encode(int,10)
         ::encode(Node_t*,0x0008)

   (pointer table: {0x0004})


= 1::ptrToBytes(0x0004)
   ::3::symbolToBytes(Node_t)
      ::2::intToBytes(2) /*two fields*/   
         ::0::intToBytes(10)
         ::1::ptrToBytes(0x0008)
            ::3::symbolToBytes(Node_t)
               ::encode(getTypeFromTable(Node_t),dereference(0x0008))

   (pointer table: {0x0004, 0x0008})


= 1::ptrToBytes(0x0004)
   ::3::symbolToBytes(Node_t)
      ::2::intToBytes(2) /*two fields*/   
         ::0::intToBytes(10)
         ::1::ptrToBytes(0x0008)
            ::3::symbolToBytes(Node_t)
               ::encode(struct Node{int, struct Node*}, {20,0x0004})
               
   (pointer table: {0x0004, 0x0008})




= 1::ptrToBytes(0x0004)
   ::3::symbolToBytes(Node_t)
      ::2::intToBytes(2) /*two fields*/   
         ::0::intToBytes(10)
         ::1::ptrToBytes(0x0008)
            ::3::symbolToBytes(Node_t)
               ::2::intToBytes(2) /*two fields*/
                  ::0::intToBytes(20)
                  ::1::ptrToBytes(0x0004)
                  ...
                  (0x0004 is in pointer table already, stop here)
 
    (pointer table: {0x0004, 0x0008})                          
\end{lstlisting}
 
0x0004 appears again, which means there must be a circle, to remember all pointer values we need an extra data structure for pointer storage and comparison.              


The decoding process for the generated \texttt{bytes[] lst} can be illustrated as follows:

\begin{lstlisting}
decode(bytes[] lst)
= decode(1::ptrToBytes(0x0004)
          ::3::symbolToBytes(Node_t)
             ::2::intToBytes(2) 
                ::0::intToBytes(10)
                ::1::ptrToBytes(0x0008)
                   ::3::symbolToBytes(Node_t)
                      ::2::intToBytes(2)
                         ::0::intToBytes(20)
                         ::1::ptrToBytes(0x0004))

= decode(3::symbolToBytes(Node_t)
          ::2::intToBytes(2) 
             ::0::intToBytes(10)
             ::1::ptrToBytes(0x0008)
                ::3::symbolToBytes(Node_t)
                   ::2::intToBytes(2)
                      ::0::intToBytes(20)
                      ::1::ptrToBytes(0x0004)) in ((t1*, 0x0004), null)                                         

= decode(2::intToBytes(2) 
          ::0::intToBytes(10)
          ::1::ptrToBytes(0x0008)
             ::3::symbolToBytes(Node_t)
                ::2::intToBytes(2) /*two fields*/
                   ::0::intToBytes(20)
                   ::1::ptrToBytes(0x0004)) in ((Node_t*, 0x0004), null)

\end{lstlisting}

Once we have the name \texttt{Node\_t}, we can directly look up and retrieve its 
associative struct type in the mapping table, and then restore a new list
on the receiver side like:

\begin{lstlisting}
Node_t* head = (Node_t*) malloc(sizeof(Node_t)); 
\end{lstlisting}

During the left decoding process, we set up a table(see table 3) which records the values on both sender and receiver sides for each pointer, to help us conveniently restore the sender side point-to relationships in the receiver side. 

\begin{table}[ht]
% title of Table
\centering 
% used for centering table
\begin{tabular}{c c c c}
% centered columns (4 columns)
\hline                        %inserts double horizontal lines
pointer & value in sender & value in receiver\\[0.5ex]% inserts table 
%heading
\hline                  % inserts single horizontal line
head & 0x0004 & 0x0012\\% inserting body of the table
tail & 0x0008 & 0x0016\\
head$\rightarrow$next & 0x0008 & 0x0016\\
tail$\rightarrow$next & 0x0004 & 0x0012\\
... & ... & ...\\


\hline
%inserts single line
\end{tabular}
\label{table:nonlin}% is used to refer this table in the text
\caption{Pointer values in both sender and receiver}
\end{table}

For example, assume in the receiver side pointer \texttt{head} equals to 0x0012, and its sender counterpart equals to 0x0004, then we have an entry like head--0x0004--0x0012 in our table. After we finished the restoration, we can also use this table to check whether the old pointer-to relationships are maintained correctly.\\

Let's continue decoding the bytes above, we have:

\begin{lstlisting}
= decode(2::intToBytes(2) 
          ::0::intToBytes(10)
          ::1::ptrToBytes(0x0008)
             ::3::symbolToBytes(Node_t)
                ::2::intToBytes(2) 
                   ::0::intToBytes(20)
                   ::1::ptrToBytes(0x0004))

= decode(3::symbolToBytes(Node_t)
          ::2::intToBytes(2) 
             ::0::intToBytes(20)
             ::1::ptrToBytes(0x0004))
  head->val = 10;
  head->next = (Node_t*)malloc(sizeof(Node_t)); //assume head->next = 0x0016 
  ("Node_t" can be directly retrieved from the name-type mapping table)
  
  Pointer table: 
  Pointer      sender   receiver
  head         0x0004    0x0012
  head->next   0x0008    0x0016

= decode(empty)

  head->next->val = 20;
  head->next->next = (Node_t*)malloc(sizeof(Node_t));//assume 0x0020
  
  Pointer table: 
  Pointer           sender   receiver
  head              0x0004    0x0012
  head->next        0x0008    0x0016
  head->next->next  0x0004    0x0020(wrong value!)
\end{lstlisting}
  
By checking the pointer table we know pointer \texttt{"head$\rightarrow$next$\rightarrow$next"} should be an alias of pointer \texttt{"head"}. So the new allocated value for \texttt{"head$\rightarrow$next$\rightarrow$next"} should be updated immediately from 0x0020 to 0x0012 as follows:

\begin{lstlisting}

= decode(empty)

  head->next->val = 20;
  head->next->next = 0x0012; 
  or 
  head->next->next = head;
  
  Pointer table: 
  Pointer           sender   receiver
  head              0x0004    0x0012
  head->next        0x0008    0x0016
  head->next->next  0x0004    0x0012
\end{lstlisting}





\section {Program dependence graph}

Building the program dependence graph(PDG) is a key step in our paper. 



Our separation is fully based on the partitioning of the program dependence graph(PDG) generated for each input program. 


 
 
\end{document}







%\begin{center}
%    \begin{tabular}{ | l | l | l | p{8cm} |}
%    \hline
%    Index & Name & Value & Description \\ \hline
%    0 & bufferLength & n$\in \mathbb{N}$ &  The buffer length used for representing the current type\\ \hline    
%    1 & isPointerTy & 1/0 & If this is a pointer type, assign 1; otherwise assign 0.  \\ \hline
%    2 & isStructTy & 1/0 & If this is a struct type, assign 1; otherwise assign 0. \\ \hline    
%    3 & isFloatTy & 1/0 & If this is a float type, assign 1; otherwise assign 0. \\ \hline    
%    4 & isIntTy & 1/0 & If this is a int type, assign 1; otherwise assign 0. \\ \hline    
%    5 & fieldNum & n$\in \mathbb{N}$ &  If the current type is a struct, fieldNum = number of fields in this struct; otherwise, leave it to zero\\ \hline    
%%    6 & value & n$\in \mathbb{N}$ &  The corresponding value for current type in program\\ \hline    
%
%    \end{tabular}
%\end{center}


%Definition decode: int[] list[0,1,...,n] -> type
%  match list[0,1,...,n] with
%  | [6][0][0][0][1][0] -> int     
%  | [6][0][0][1][0][0] -> float
%  | [len][1][0][0][0][0]::list[6,...,n] -> {decode(list[6,...,n])}*   
%  | [len][0][1][0][0][n]::list[6,...,n] ->
%    struct{
%      decode(list[6,...,6+list[6]-1]);  
%      decode(list[6+list[6],...,6+list[6+list[6]]-1])};
%      ...
%      decode(list[6+list[6+...]],...,n);
%    }             
%  end.















%
%
%The basic use of the rule is 
%\begin{verbatim}
%\inferrule
%  {one \\ two \\ three \\ or \\ more \\ premisses}
%  {and \\ any \\ number \\ of \\ conclusions \\ as \\ well}
%\end{verbatim}
%This is the rendering on a large page
%\def \one {\inferrule
%  {one \\ two \\ three \\ or \\ more \\ premises}
%  {and \\ any \\ number \\ of \\ conclusions \\ as \\ well}
%}
%$$
%\ifhevea \one
%\else
%\fbox {\vbox {\advance \hsize by -2\fboxsep \advance \hsize by -2\fboxrule
%       \linewidth\hsize
%        $$\one$$}}
%\fi
%$$
%However, the same formula on a narrower page will automatically be typeset
%like that:
%$$
%\ifhevea \one
%\else
%\fbox {\hsize 0.33 \hsize \vbox {$$\one$$}}
%\fi
%$$
%An inference rule is mainly composed of a premise and a conclusion. 
%The premise and the conclusions are both list of formulas where the
%elements are separated by \verb"\\". 
%
%Note the asymmetry between typesetting of the premises and of
%conclusions where lines closer to the center are fit first. 
%
%A newline can be forced by adding an empty line \verb"\\\\"
%
%\begin{tabular}{m{0.44\hsize}m{0.44\hsize}}
%\begin{lstlisting}{Ocaml}
%\inferrule 
%   {aa \\\\ bb}
%   {dd \\ ee \\ ff}
%\end{lstlisting}
%&
%$\inferrule {aa \\\\bb}{dd \\ ee \\ ff}$
%\\
%\end{tabular}
%
%\subsection {Single rules}
%
%Single rules are the default mode. 
%Rules are aligned on their fraction bar, as illustrated below:
%$$
%\inferrule {aa \\ bb}{ee} \hspace {4em} \inferrule {aa \\\\ bb \\ ee}{ee}
%$$
%If the premise or the conclusion is empty, then the fraction bar is not
%typeset and the premise or the conclusion is centered:
%$$
%\begin{tabular}{m{0.45\hsize}m{0.45\hsize}}
%\begin{lstlisting}{Ocaml}
%\inferrule {}{aa} + 
%\inferrule {aa \\\\ aa}{}
%\end{lstlisting}
%&
%$
%\inferrule {}{aa} + 
%\inferrule {aa \\\\ aa}{}
%$
%\\
%\end{tabular}
%$$
%Use use \verb"{ }" instead of \verb"{}" to get an axiom for instance:
%$$
%\begin{tabular}{m{0.45\hsize}m{0.45\hsize}}
%\begin{lstlisting}{Ocaml}
%\inferrule { }{aa} + 
%\inferrule {aa}{ }
%\end{lstlisting}
%&
%\mbox {$
%\inferrule { }{aa} + 
%\inferrule {aa}{ }
%$}
%\\
%\end{tabular}
%$$
%
%The macro \lst"\inferrule" accepts a label as optional argument, which will
%be typeset on the top left corner of the rule: 
%\par
%\begin{tabular}{m{0.45\hsize}m{0.45\hsize}}
%\begin{lstlisting}{Ocaml}
%\inferrule [yop]
%   {aa \\ bb}
%   {cc}
%\end{lstlisting}
%&
%$\inferrule [Yop]{aa \\ bb}{cc}$
%\\
%\end{tabular}
%\par\noindent
%See section~\ref {options} for changing typesetting of labels. 
%A label can also be placed next to the rule directly, since the rule is
%centered: 
%\par
%\begin{tabular}{m{0.45\hsize}m{0.45\hsize}}
%\begin{lstlisting}{Ocaml}
%\inferrule 
%   {aa \\ bb}
%   {cc}
%\quad (\textsc {Yop})
%\end{lstlisting}
%&
%$\inferrule{aa \\ bb}{cc} \quad (\textsc {Yop})$
%\\
%\end{tabular}
%
%\subsection {Customizing presentation}
%
%By default, lines are centered in inference rules. 
%However, this can be changed by either \lst"\mprset{flushleft}"
%or \lst"\mprset{center}". For instance, 
%
%\begin{tabular}{m{0.44\hsize}m{0.44\hsize}}
%\begin{lstlisting}{Ocaml}
%$$\mprset{flushleft}
%  \inferrule 
%    {a \\ bbb \\\\ ccc \\ dd}
%    {dd \\ ee \\ ff}$$ 
%\end{lstlisting}
%&
%$$\mprset{flushleft} 
%\hsize 0.45\hsize
%\inferrule {a \\ bb  \\ ccc \\ dddd}{e \\ ff \\ gg}$$
%\\
%\end{tabular}
%
%\noindent
%Note that lines are aligned independently in the premise and the
%conclusion, which are both themselves centered. In particular, 
%left alignment will not affect a single-line premise or conclusion. 
%
%\subsection {Customizing rules}
%
%One may wish to change use rules for rewriting rule or implications, etc.
%There is a generic way of definition new rules by providing three parts: 
%a tail, a body, and a head. The rule will then be built by joining
%all three components in this order and filling the body with leaders to
%extend as much as necessary. Here are examples
%
%\begin{tabular}{m{0.54\hsize}m{0.44\hsize}}
%\begin{lstlisting}{Ocaml}
%$$\mprset{fraction={===}}
%  \inferrule {a \\ bbb} {cc}$$ 
%\end{lstlisting}
%&
%$$\mprset{fraction={===}}
%  \inferrule {a \\ bbb} {cc}$$ 
%
%\\
%\begin{lstlisting}{Ocaml}
%$$\mprset
%  {fraction={\models=\Rightarrow}}
% \inferrule {a \\ bbb} {cc}$$ 
%\end{lstlisting}
%&
%$$\mprset
%  {fraction={\models=\Rightarrow}}
%  \inferrule {a \\ bbb} {cc}$$ 
%\\
%\end{tabular}
%The height and depth of the \emph{body} are used to adjust vertical space. 
%One, may ``smash'' the body to reduce the vertical space
%
%\begin{tabular}{m{0.54\hsize}m{0.44\hsize}}
%\begin{lstlisting}{Ocaml}
%$$\mprset
%  {fraction={%
%      {\scriptstyle\vdash}%
%      {\smash-}%
%      {\rightarrow\!\!}%
%      }}
% \inferrule {a \\ bbb} {cc}\,\,$$ 
%\end{lstlisting}
%&
%$$\mprset
%  {fraction={{\scriptstyle\vdash}{\smash-}{\rightarrow\!\!}}}
%  \inferrule {a \\ bbb} {cc}$$ 
%\\
%\begin{lstlisting}{Ocaml}
%$$\mprset {fraction={\cdot\cdots\cdot}}
% \inferrule {a \\ bbb} {cc}$$ 
%\end{lstlisting}
%&
%$$\mprset {fraction={{}{\,\smash\cdot\,}{}}}
%  \inferrule {a \\ bbb} {cc}$$ 
%\\
%\end{tabular}
%Since vertical skip does not take header and footer into account, which is
%usually better but sometimes odd, this can be adjusted explicitly:
%
%\begin{tabular}{m{0.54\hsize}m{0.44\hsize}}
%\begin{lstlisting}{Ocaml}
%$$\mprset
%  {fraction={|=/},
%   fractionaboveskip=0.6ex,
%   fractionbelowskip=0.4ex}
% \inferrule 
%   {a \\ bbb_{\downarrow}} 
%   {cc^{\T\uparrow}}$$ 
%\end{lstlisting}
%&
%$$\mprset
%  {fraction={|=/},
%   fractionaboveskip=0.6ex,
%   fractionbelowskip=0.4ex}
% \inferrule 
%   {a \\ bbb_{\downarrow}} 
%   {cc^{\uparrow}}$$ 
%\\
%\end{tabular}
%Finally, it is also possible to provide its own definition 
%of fraction by
%
%\begin{tabular}{m{0.54\hsize}m{0.44\hsize}}
%\begin{lstlisting}{Ocaml}
%\def \Over #1#2{\hbox{$#1 \over #2$}}
%$$\mprset{myfraction=\Over}
%  \inferrule {a \\ bbb} {cc}$$ 
%\end{lstlisting}
%&
%\def \Over #1#2{\hbox{$#1 \over #2$}}
%$$\mprset{myfraction=\Over}
%  \inferrule {a \\ bbb} {cc}$$ 
%\\
%\end{tabular}
%
%\paragraph{Customizing the horizontal skip between premises} 
%(default value is 2em). 
%\begin{quote} 
%\begin{lstlisting}{Ocaml} 
%$$\mprset {sep=6em} 
% \inferrule {a \\ bbb} {cc}$$ 
%\end{lstlisting} 
%$$\mprset {sep=6em} 
%  \inferrule {a \\ bbb} {cc}$$ 
%\end{quote}                  
%\paragraph{Customizing the vertical space between premises}
%
%(default value is empty).  Notice that leaving it empty and setting vskip to
%0em is not quite equivalent as show below between the third and fourth rules
%(because the typesetting cannot use the primitive typesetting of
%fractions). 
%\begin{quote}
%\begin{lstlisting}{Ocaml}
%$$\def\R{\inferrule {aa\\aa\\\\bbb\\bbb} {cc}
%  \hspace{3em}}
%  \R  \mprset{vskip=0ex}\R  \mprset{vskip=1ex}\R$$
%\end{lstlisting}
%$$\def \R{\inferrule {aa \\ aa  \\\\ bbb \\ bbb} {cc}\hspace{3em}}
%  \R  \mprset{vskip={}}
%  \R  \mprset{vskip=0ex}\R  \mprset{vskip=1ex}\R$$
%\end{quote}
%
%
%
%\subsection {Tabulars in inference rules}
%
%Although you probably do not want to do that, you may still use tabular
%or minipages inside inference rules, but between braces, as follows:
%$$
%\begin{tabular}{m{0.50\hsize}m{0.50\hsize}}
%\begin{lstlisting}{Ocaml}
%\infer [Tabular-Rule]
%{some \\ math \\ and \\
% {\begin{tabular}[b]{|l|r|}
% \hline Ugly & and
%  \\[1ex]\hline 
%  table & text
%  \\\hline
% \end{tabular}} \\
% {\begin{minipage}[b]{6em}
%  Do you really wish
%  to do that? 
%  \end{minipage}} \\ 
%}
%{some \\ conclusions}
%\end{lstlisting}
%&
%\infer [Tabular-Rule]
%{some \\ math \\ and \\
% {\begin{tabular}[b]{|l|r|}
% \hline Ugly & and
%  \\[1ex]\hline
%  table & text
%  \\\hline
% \end{tabular}} \\
% {\begin{minipage}[b]{6em}
%  Do you really wish
%  to do that? 
%  \end{minipage}} \\ 
%}
%{some \\ conclusions}
%\\
%\end{tabular}
%$$
%
%
%\subsection {Derivation trees}
%
%To help writing cascades of rules forming a derivation tree, inference rules
%can also be aligned on their bottom line. For this, we use the star-version:
%$$
%\begin{tabular}{m{0.65\hsize}m{0.45\hsize}}
%\begin{lstlisting}{Ocaml}
%\inferrule*
%   {\inferrule* {aa \\ bb}{cc}
%    \\ dd}
%   {ee}
%\end{lstlisting}
%&
%$
%\inferrule*
%   {\inferrule* {aa \\ bb}{cc}
%    \\ dd}
%   {ee}
%$
%\\
%\end{tabular}
%$$
%The star version can also take an optional argument, 
%but with a different semantics. The optional argument is parsed by the 
%\verb"keyval" package, so as to offer a set of record-like options:
%$$
%\def \arraystretch {1.4}
%\begin{tabular}{|>{\tt}c|>{$}c<{$}|p{0.6\hsize}|}
%\hline
%\bf key & \bf arg & \bf Effect
%\\\hline
%before & tex & Execute $tex$ before typesetting the rule.
%         Useful for instance to change the maximal width of the rule.
%\\\hline
%width & d &  Set the width of the rule to $d$
%\\\hline
%narrower & d & Set the width of the rule to $d$ times \verb"\hsize".
%\\\hline
%lab & \ell & Put label $\ell$ on  the top of the rule as with the
%non-start version.
%\\\hline
%Lab & \ell & same as lab
%\\\hline
%left & \ell & Put label $\ell$ on the left of the rule
%\\\hline
%Left & \ell & Idem, but as if label $\ell$ had zero width. 
%\\\hline
%Right & \ell & As \verb"Left", but on  the right of the rule.
%\\\hline
%right & \ell & As \verb"left",  but on the right of the rule.
%\\\hline
%leftskip & d & Cheat by (skip negative space) $d$ on the left side.
%\\\hline
%rightskip & d & Cheat by $d$ on the right side of the rule.
%\\\hline
%vdots & d & Raise the rule by $d$ and insert vertical dots. 
%\\\hline
%\end{tabular}
%$$
%
%We remind at the end the global options that we've seen above that can
%also be set locally in derivation trees: 
%\begin{mathpar}
%\def \arraystretch {1.4}
%\begin{tabular}{|>{\tt}c|>{$}c<{$}|p{0.6\hsize}|}
%\hline
%\\\hline\hline
%sep & d & Set the separation between premises and conclusions to $s$.
%\\\hline
%flushleft & - & flush premises to the left hand side
%\\\hline
%center & - & center premises on each line.
%\\\hline
%rewrite & d & 
%\\\hline
%myfraction & tex & set fraction to $tex$ command
%\\\hline
%fraction & lmr & set fraction pattern to $lm...mr$ with leaders. 
%\\\hline
%vskip & d & Set the vertical skip between premises and conclusions to $h$. 
%\\\hline
%vcenter && Make the rule centered around the fraction line as the non-star
%version
%\\\hline
%\end{tabular}
%\end{mathpar}
%
%Here is an example of a complex derivation:
%$$
%\inferrule* [left=Total,rightstyle=\em,right={(when $n > 0$)}]
%  {\inferrule* [Left=Foo]
%     {\inferrule* [rightstyle={\bf},Right=Bar,vskip=1ex,
%                   leftskip=2em,rightskip=2em,vdots=1.5em]
%         {a \\ a \\\\ bb \\ cc \\ dd}
%         {ee}
%       \\ ff \\ gg}
%     {hh}
%  \\
%  \inferrule* [lab=XX,sep=4em]{uu \\ vv}{ww}}
%  {(1)}
%$$
%and its code
%\begin{lstlisting}{Ocaml}
%\inferrule* [left=Total]
%  {\inferrule* [Left=Foo]
%     {\inferrule* [Right=Bar,rightstyle=\bf,
%                       leftskip=2em,rightskip=2em,vdots=1.5em]
%         {a \\ a \\\\ bb \\ cc \\ dd}
%         {ee}
%       \\ ff \\ gg}
%     {hh}
%  \\
%  \inferrule* [lab=XX]{uu \\ vv}{ww}}
%  {(1)}
%\end{lstlisting}
%
%\def \L#1{\lower 0.4ex \hbox {#1}}
%\def \R#1{\raise 0.4ex \hbox {#1}}
%\def \hevea {H\L{E}\R{V}\L{E}A}
%\def \hevea {$\mbox {H}\!_{\mbox {E}}\!\mbox {V}\!_{\mbox {E}}\!\mbox {A}$}
%
%\subsection {Label styles}
%
%\label {options}
%
%The package uses
%\verb"\DefTirNameStyle",
%\verb"\LabTirNameStyle",
%\verb"\LeftTirNameStyle",
%and \verb"\RightTirNameStyle"
%to typeset labels introduced with the default option,
%\verb"Lab-",
%\verb"Left-", or
%\verb"Right-", respectively (or their uncapitablized variants). 
%This can safely be redefined by the user.
%\verb"\DefTirName" is normally used for defining
%occurrences ({\em i.e.} in rule \lst"\inferrule") while the three other forms
%are used for referencing names ({\em i.e.} in the star-version).  The styles
%can also be redefined using labeled-arguments of the star-version of
%{\tt\string\inferrule} as described in table below.
%
%Instead of just changing the style, the whole
%typesetting of labels may be changed by redefining
%\verb"\DefTirName",
%\verb"\LabTirName",
%\verb"\LeftTirName",
%and \verb"\RightTirName", each of which receives the label to be typeset as
%argument.
%
%Finally, the vertical skip
%
%\begin{mathpar}
%\def \arraystretch {1.4}
%\begin{tabular}{|>{\tt}c|>{$}c<{$}|p{0.6\hsize}|}
%\hline
%\bf key & \bf arg & \bf Effect
%\\\hline\hline
%style & tex & set the default style for labels to $tex$
%\\\hline
%leftstyle & tex & idem for  labels
%\\\hline
%rightstyle & tex& idem for right  labels
%\\\hline
%\end{tabular}
%\end{mathpar}
%  
%
%\subsection {Star \emph{v.s.} non-star version}
%
%The package also defines \verb"\infer" as a shortcut for \verb"\inferrule"
%but only if it is not previously defined.
%
%There are two differences between the plain and star versions of
%\verb"\inferrule".
%The plain version centers the rule on the fraction line, while the
%star one centers the rule on the last conclusion, so as to be used in
%derivation trees.
%
%Another difference is that the optional argument of the plain version is a
%label to always be placed on top of the rule, while the $\ast$-version takes
%a record of arguments.  Hence, it can be parameterized in many more ways.
%
%One may recover the plain version from the start version by passing the
%extra argument \texttt{vcenter} as illustrated below (the base line is
%aligned with the dotted line):
%\begin{mathpar}
%\cdots\cdots
%\cdots\cdots
%\inferrule* 
%      {aa  aa \\\\ aa \\ bb}{cc \\ cc \\\\ dd}
%\cdots\cdots
%\cdots\cdots
%\inferrule* [vcenter]
%      {aa  aa \\\\ aa \\ bb}{cc \\ cc \\\\ dd}
%\cdots\cdots
%\cdots\cdots
%\end{mathpar}
%This is convenient, for instance to typeset rules with side conditions
%and keep them attached to the rule:
%\begin{mathpar}
%\def \RightTirName #1{\rm\hbox {\hskip 1ex (#1)}}
%\inferrule*[lab=Pos,right={if $n>0$}]
%        {aa \\  aa}
%        {cc}
%
%\inferrule*[lab=Neg,right={if $n<0$}]
%        {aa \\  aa}
%        {cc}
%\end{mathpar}
%Or differently, 
%\begin{mathpar}
%\def \LabTirName #1{\hbox {(#1)}}
%\def \LeftTirName #1{\textsc{#1}}
%\inferrule*[Left=Pos,lab={if $n>0$}]
%        {aaa \\  aaa}
%        {cc}
%
%\inferrule*[Left=Neg,lab={if $n<0$}]
%        {aaa \\  aaa}
%        {cc}
%\end{mathpar}
%
%
%\subsection {Implementation}
%
%The main macro in the implementation of inference rules is the one that
%either premises and conclusions.  The macros uses two box-registers one
%\verb"hbox" for typesetting each line and one \verb"vbox" for collecting
%lines. The premise appears as a list with
%\verb"\\" as separator. Each element is considered in turn typeset in a
%\verb"hbox" in display math mode. Its width is compare to the space left on
%the current line. If the box would not fit, the current horizontal line is
%transferred to the vertical box and emptied. Then, the current formula can
%safely be added to the horizontal line (if it does not fit, nothing can be
%done). When moved to the vertical list, lines are aligned on their center
%(as if their left-part was a left overlapped). At the end the vbox is
%readjusted on the right. 
%
%This description works for conclusions. For premises, the elements must be
%processes in reverse order and the vertical list is simply built upside
%down. 
%
%\section {Other Options for the {\tt mathpar} environment}
%
%The vertical space in \verb"mathpar" is adjusted by
%\verb"\MathparLineskip". To restore the normal paragraph parameters in mathpar
%mode (for instance for some inner paragraph), use the command
%\verb"\MathparNormalpar".  
%The environment uses \verb"\MathparBindings" to
%rebind \verb"\\", \verb"and", and \verb"\par". You can redefine thus command
%to change the default bindings or add your own. 
%
%
%\section {Examples}
%
%See the source of this documentation ---the file \lst"mathpartir.tex"---
%for full examples. 
%
%\section {{\hevea} compatibility}
%
%The package also redefines \verb"\hva" to do nothing in \lst"mathpar"
%environment and nor in inference rules. 
%
%In HeVeA, \verb"\and" will always produce a vertical break in mathpar
%environment; to obtain a horizontal break, use \verb"\hva \and" instead.
%Conversely, \verb"\\" will always produce a horizontal break in type
%inference rules; to obtain a vertical break, use \verb"\hva \\" instead.
%
%For instance, by default the following code,
%\begin{lstlisting}{Ocaml}
%\begin{mathpar}
%\inferrule* [Left=Foo]
%   {\inferrule* [Right=Bar,width=8em,
%                 leftskip=2em,rightskip=2em,vdots=1.5em]
%      {a \\ a \\ bb \\ cc \\ dd}
%      {ee}
%    \\ ff \\ gg}
%   {hh}
%\and
%\inferrule* [lab=XX]{uu \\ vv}{ww}
%\end{mathpar}
%\end{lstlisting}
%which typesets in {\TeX} as follows,
%\begin{mathpar}
%\inferrule* [Left=Foo]
%   {\inferrule* [Right=Bar,width=8em,
%                 leftskip=2em,rightskip=2em,vdots=1.5em]
%      {a \\ a \\ bb \\ cc \\ dd}
%      {ee}
%    \\ ff \\ gg}
%   {hh}
%\and
%\inferrule* [lab=XX]{uu \\ vv}{ww}
%\end{mathpar}
%would appear as follows with the compatible {\hevea} mode:
%\begin{mathpar}
%\inferrule* [left=Foo]
%   {\inferrule* [right=Bar]
%      {a \\ a \\ bb \\ cc \\ dd}
%      {ee}
%    \\ ff \\ gg}
%   {hh}
%\\
%\inferrule* [lab=XX]{uu \\ vv}{ww}
%\end{mathpar}
%To obtain (almost) the same rendering as in {\TeX}, it could be typed as 
%\begin{lstlisting}[escapechar=\%]{Ocaml}
%  \begin{mathpar}
%      \inferrule* [Left=Foo]
%      {\inferrule* [Right=Bar,width=8em,
%                    leftskip=2em,rightskip=2em,vdots=1.5em]
%         {a \\ a \hva \\ bb \\ cc \\ dd}
%         {ee}
%         \\ ff \\ gg}
%      {hh}
%      \hva \and
%  \inferrule* [lab=XX]{uu \\ vv}{ww}
%  \end{mathpar}
%  \end{lstlisting}
%  Actually, it would be typeset and follows with the compatible {\hevea} mode:
%  \begin{mathpar}
%      \inferrule* [left=Foo]
%      {\inferrule* [right=Bar]
%         {a \\ a \\\\ bb \\ cc \\ dd}
%         {ee}
%         \\ ff \\ gg}
%      {hh}
%      \and
%  \inferrule* [lab=XX]{uu \\ vv}{ww}
%\end{mathpar}
%  
%
%
%  
%\end{document}
%  
%  
%% LocalWords:  mathpar aa Yop bb dd ee ff cc Ocaml Foo leftskip rightskip vdots
%% LocalWords:  gg hh uu vv ww HeVeA escapechar inferrule yop flushleft bbb
%  
%%%  LocalWords:  ccc dddd fractionaboveskip fractionbelowskip myfraction
%%%  LocalWords:  sep vskip Tabulars minipages keyval arg tex lmr vcenter
%%%  LocalWords:  rightstyle uncapitablized leftstyle parameterized Pos aaa
%%%  LocalWords:  hbox vbox mathpartir
%
%
%
%
%
%
%%% trash
%
%
%
%This package provides macros for displaying lists of formulas that are
%typeset in mixed horizontal and vertical modes. The package is two-folded.
%
%The first part is an environment \verb"mathpar" that generalizes the math
%display mode to allow several formulas on the same line, and several lines
%in the same display. The arrangement of the sequence of formulas into lines
%is automatic depending on the line width and on a minimum inter-formula
%space and line width alike words in a paragraphs (in centerline mode).  A
%typical application is  displaying a set of type inference rules.
%
%The second par is a macro \lst"inferrule" to typeset inference rules
%themselves.  Here again, both premises and conclusions are presented as list
%of formulas that should be displayed in almost the same way, except that the
%width is not fixed in advance; and the inference rule should use no more
%width than necessary so that other inference rules are given a chance to
%appear on the same line.
%
%Although \verb"mathpar" and \verb"inferrule" look similar in their
%specification, and are often used in combination, they are in fact
%completely different in their implementations.
%
%
%
%\ifhevea
% \section {License}
%
%Mathpartir is Copyright (C) 2001, 2002, 2003, 2005, 2015 INRIA.  Mathpartir
%has been developed by Didier R{\'{e}}my.  Mathpartir is free software; you
%can redistribute it and/or modify it under the terms of the GNU General
%Public License as published by the Free Software Foundation; either version
%2, or (at your option) any later version.  See the GNU General Public
%License for more details (\url{http://pauillac.inria.fr/~remy/license/GPL}).
%Mathpartir is distributed in the hope that it will be useful, but without
%any warranty.
%\fi
%
%
%without a powerful protocol
%
%If we only need to deal with scalar types and simple composite types like this, there is no need to design . 
%
%$$
%This is Courier font
%$$
%
%
%The mathpar environment is a ``paragraph mode for formulas''. 
%It  allows to typeset long list of formulas putting as
%many as possible on the same line: 
%
%
%$$
%However, the same formula on a narrower page will automatically be typeset
%like that:
%$$
%
%
%
%
%
%$$
%\begin{tabular}{m{0.45\hsize}m{0.45\hsize}}
%\begin{lstlisting}{Ocaml}
%\begin{mathpar}
%A-Formula \and 
%Longer-Formula \and 
%And \and The-Last-One
%\end{mathpar}
%\end{lstlisting}
%&
%\begin{mathpar}
%A-Formula 
%\and
%Longer-Formula
%\and
%And 
%\and
%The-Last-One
%\end{mathpar}
%\end{tabular}
%$$
%Formulas are separated by \verb"\and" (or equivalently by a blank line). 
%To enforce a vertical break it suffices to replace \verb"\and" by
%\verb"\\". 
%
%The implementation of \verb"mathpar" entirely relies on the paragraph mode
%for text. It starts  a new paragraph, and a math formula within a paragraph,
%after adjusting the spacing and penalties for breaks. Then, it simply binds
%\verb"\and" to something like \verb"\goodbreak". 
%
%
%
%
%
%
